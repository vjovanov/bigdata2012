Cluster computing systems today impose a trade-off between generality,
performance and productivity. Hadoop and Dryad force programmers to write low
level programs that are tedious to compose but easy to optimize. Systems like
Dryad/LINQ and Spark allow concise modeling of user programs but do not apply
relational optimizations. Pig and Hive restrict the language to achieve
relational optimizations, making complex programs hard to express without user
extensions. However, these extensions are cumbersome to write and disallow
program optimizations.

We present a distributed batch data processing framework called \tool.
\tool uses deep language embedding in Scala, multi-stage programming and explicit side effect
tracking to analyze the structure of user programs. The analysis is used to
apply projection insertion, which eliminates unused data, as well as code
motion and operation fusion to highly optimize the performance critical path of
the program. The language embedding and a high-level interface allow \tool
programs to be both expressive, resembling regular Scala code, and optimized.
Modular design allows users to extend \tool with modules, which specify an
abstract interface and how to generate high performance code for it. Through a
modular code generation scheme, \tool can execute programs on both Spark and
Hadoop. Compared with na\"{\i}ve implementations we achieve 143\% speedups on
Spark and 126\% on Hadoop.
