\section{Background}
\label{sec:background}

% language virtualization
\subsection{Virtualized Scala}
\label{subsec:virtualized-scala}
\tool is written in an experimental version of Scala called Virtualized Scala \cite{sv}. Virtualized Scala provides facilities for simplified deep embedding of domain specific languages. Main facility for achieving this is the desugaring of regular language constructs like conditionals, loops, variable declarations, return statements and equality comparisons to simple method calls. For example, \code{if (c) a else b} gets translated to an \code{__ifThenElse(c, a, b)} method call. In the case of regular language execution this method executes the logic of the conditional but in case of deeply embedded DSLs it is used for creation of  IR node that represents the \code{if} statement.  

In Virtualized Scala, all embedded DSLs are written within DSL scopes. These special scopes look like method invocations that take one by name parameter. However, they get translated to the complete specification of DSL modules that are used in a form of a Scala trait mix-in composition\todo{put a star what is a mixin composition}. For example: 
\code{stivo\{ \\\\ dsl code \}} gets translated into:
\scode{new StivoDSL \{
 def main()\{
   \\\\ dsl code
 \}
\}} 
This makes all the DSL functionality defined in StivoDSL visible in body of the by name parameter passed to stivo method. 
Although modified, Virtualized Scala is fully binary compatible with Scala and can use all existing libraries.  

  
%??? infix-methods, if the type of expression x does not have method m but there is an infix method defined in the scope it will be desugared to infix_m(x) of the expression x m does not type check 

\subsection{Lightweight Modular Staging}
\label{subsec:lightweight-modular-staging}

% lms basics 
The base for the \tool project is LMS (Lightweight Modular Staging) library \todo{cite}. It utilizes facilities provided by Virtualized Scala to build a modular infrastructure for developing staged DSLs. LMS library represents types in a DSLs with an polymorphic abstract data type \code{Rep[T]}. When in a DSL scope term has type \code{Rep[T]} it means that once the code is staged, optimized and generated the actual result of the term will have type \code{T}.  Since \code{Rep[T]} is and abstract type each DSL module can specify concrete operations on it which are used for simple building of the DSL intermediate representation. For example: 

% Here we represent a simple Exp example in several lines of code.
 
Since Scala's type system supports type inference \todo{which? It is not HM.} most of the Rep[T] types are hidden from the DSL user. This makes the DSL code almost completely free of type information and makes the user almost unaware of the Rep types. The only exception for visibility of Rep types are the parameters of defined methods and fields of defined classes. In our experience with writing DSLs, Rep types do not present a problem but gathering precise and unbiased information is very difficult. 

% LMS effects
Unlike Scala, which does not have an official effects tracking mechanism, LMS expects the DSL developer to explicitly specify exact effects for each DSL operation. LMS effect tracking system than calculates the effects summary for each basic block in the code. This allows the optimizer to apply code motion operations on th e pure parts of the code and remove constant expressions out of hot loops as well as push expressions inside conditionals. The framework provides effects tracking for numerical operations and all basic language constructs, but all other effects are up to the DSL developer. Although, it is hard for the DSL developer to specify effects for each operation the overall performance gain is large. Also, due to LMS modularity the effects for most commonly used facilities like strings, arrays, loops and conditionals are 100\% reusable.


% modularity through scalable components abstraction.
Modular design of LMS allows the DSL developer to arbitrary compose the interface, optimizations and code generation of the DSL. Since LMS provides modules for large part of the Scala language and most common libraries if the DSL developer decides to include all the modules, the language will be very general and similar to the native Scala language. Module inclusion is simply done by mixing in Scala traits together. The correctness of the composition are checked by the type system and missing dependencies are caught by the type checking system. Code generation for a DSL is also modular so actual module for a single DSL is usually very thin since it reuses other modules in the library.