\section{Discussion}
\label{sec:discussion}
 
% Generality of our approach
The language we provide is the same as Scala in its basic constructs, however it
does not support all of the functionalities. The following functionalities are
not available:
\begin{itemize}
\item Subtype polymorphism (for user defined classes) is currently not
supported.
\item \tool can only apply optimizations on functions provided as a DSL module
for LMS.
\end{itemize}

% Although it is not completely general   


% Code compilation and generation vs interactivity
One of the caveats of the staged DSL approach is that the program staging,
compilation, generation and compilation of the generated code increases the
startup time of the task. For the benchmarks we tested this process takes
between 4 and 14 seconds. Although this can be significant, it needs
to be done only once on a single machine so we believe it is not a limiting
factor for batch jobs.

The only case where compile time becomes relevant is with back-ends that support
interactive data analytics, like the Spark framework. Spending more than a
couple of seconds for compilation would affect the interactivity.

In cases where the data processing time is comparable to the user reaction
time, the optimizations we perform are not needed. We could implement a version
of \tool that does not perform staging but instead executes the original code
straight away.
%However, if data processing time is comparable to user reaction time,
%optimizations are not needed and we could implement a version of \tool which
% does not generate the IR, but executes the original code straight away. 
This feature is not
implemented in \tool, but Kossakowski et al. \cite{greg} have done this for a
Javascript DSL.

% Job specific tuning
Each job requires a framework-specific configuration for its optimal execution
(e.g. the number of mappers, reducers, buffer sizes etc.). Our current API does
not include tuning of such parameters, but in the future work we want to
introduce a configuration part of the DSL to unify the configuration of
different back-ends.
