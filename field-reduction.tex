\section{Optimizations}
\label{sec:optimizations}
In this section we present the projection insertion and the operation fusion
optimizations implemented in \tool.
\subsection{Projection Insertion}
\label{sec:field-reduction}
\newcommand{\aos}{AoS $\rightarrow$ SoA }

% Intro/Motivation
A common optimization in data processing is to remove intermediate values early
that are not needed in later phases of the computation. It has been implemented
in relational databases for a long time, and has recently been added to the Pig
framework. This optimization requires all field accesses in the program to be
explicit. A library can provide this, but its usage is more intrusive than if
the framework can use compiler support.

% description of supported types
In \tool, we support this optimization for algebraic data types, more
specifically final immutable Scala classes with a finite level of nesting. Our
approach does not require special syntax or access operators and supports method
declarations on data types just like methods of regular Scala classes. While
implementing our benchmarks we found this to be a reasonably expressive model
for big data programming. The DSL user needs to supply class declarations, from
which we generate all the necessary code for its use in \tool.
% In these cases, LMS describes all field accesses explicitly and we can
% generate highly specialized code for these types serialization schemes for the
% backends we support.

% Explain why our algorithm is so simple
A projection insertion optimization needs to know about the liveness of all
fields it can possibly remove. \aos optimization in LMS provides this within a
scope, by decomposing all control structures to multiple copies of it accessing
each field separately. After DCE, all remaining fields are alive within that
scope. For \tool, this is not enough, as we support operations in the data-flow
graph, like \code{groupByKey}, for which we found no good way to decompose.
However, we can define a liveness analysis for each operation in our programming
model. For all operations in our programming model we identify rules on how this
operation influences the liveness of fields. For operations that have a closure
parameter, we apply \aos inside that closure to acquire liveness information.

By performing this analysis on each node in reverse topological order and
propagating the liveness information to its predecessors, we are able to perform
removal of unused fields in all operations. On a distributed program, the
removal of dead fields is especially important before an operation that requires
network transport of an object or stores it in memory. We call such an operation
a barrier, and insert a projection which only contains the live fields before
it.
% An additional projection may introduce CPU overhead without offering a benefit
% in the case there is no constructor invocation in the predecessor's closure's
% scope. In these cases we simply do not insert a projection.

% introduce paths
Since we support nested classes of a finite level, the nested fields of a class
form a tree, and if a field in such a tree is alive, it requires liveness of all
its ancestors. We call the path of a nested field to the root of the tree an
\emph{access path}, and represent it using a string. The Figure
\ref{fig:type_tree} shows the tree of nested fields for the class Tuple2[String,
A]. The nodes describe the class of a parent's field, while the edges represent
the field name. The \emph{access path} to each nested field is formed by
concatenating the edges with a separating dot. In the Figure, the access path
for the field \code{id} in class \code{B} would be \code{_2.b.id}. For each edge
in the data-flow graph, that our operations form, we need to compute the set of
access paths.

\begin{figure}[b]
% \begin{subfigure}
\begin{lstlisting}[language=Scala,name=code]
case class A(id: String, b: B)
case class B(id: String)  
val t = ("tuple", A("a", B("b"))) 
t: scala.Tuple2[String, A]
\end{lstlisting}
% \end{subfigure}
% \begin{subfigure}
\centering
\includegraphics[clip=true, width=0.95\columnwidth]{dot/access.png}
\caption{Visualization of the tree of fields for class Tuple2[String, A]}
\label{fig:type_tree}
% \end{subfigure}
\end{figure}

% explaining the analysis
For each operation, we need to define how the access paths used by it are
translated into the access paths it uses. For this analysis we used following
primitives:
\begin{itemize}
\item \emph{Access paths for a type}: Given a type, this primitive creates
access paths for all the nested fields within it. In Figure \ref{fig:type_tree}
a \code{DColl.save()} with elements of class A returns the access paths
${id, b, b.id}$.
\item \emph{Closure analysis}: This primitive returns a list of all access paths
on the closure's input type.
\item \emph{Rewrite access paths}: Several operations have semantics which
influence the type and therefore the access paths. For example, the \code{cache}
operation will always have the same input and output type and is known not to
change any fields, so all access paths from its successors must be propagated to
its predecessors. The \code{groupByKey} operation on the other hand always reads
all nested fields of the key, and has to rewrite all access paths of the value.
\item \emph{Narrow closure}: Given a closure, this primitive replaces the
closure's original output with a narrowed one based on the access paths of its
successors.
\end{itemize}

% \begin{table}[width=0.5\pagewidth, float=t]
% 
%     \begin{tabularx}{0.5\textwidth}{l|X|l}
%         Operation    & Propagate access paths 							     & Barrier \\ \hline
%         \code{filter}       & All access paths of successor + access paths of closure                                                      & ~       \\ 
%         \code{flatMap}      & Return closure analysis of narrowed closure                                             & ~       \\ 
%         \code{map}          & Return closure analysis of narrowed closure                                           & ~       \\ 
%         \code{join}         & Generates access paths all nested fields of the key, and propagates access paths to the values to the correct predecessor.    & x       \\ 
%         \code{groupByKey}   & Generates access paths all nested fields of the key, propagates the accesses to the value's iterable to the value itself  & x       \\ 
%         \code{reduce}       & All accesses from the closure are translated to accesses of the value's iterable and propagated        & ~       \\ 
%         \code{save}         & Generates all access paths for the input type           	                                             & ~       \\ 
%     \end{tabularx}
%     
%     \caption{Access path computation and propagation for selected operations.}
%     \label{table:field_reduction}
% \end{table}

\newcommand{\ar}{\Rightarrow}
\newcommand{\REWRITE}[3]{{rewrite(#1, \space #2 \ar #3)}}
\newcommand{\CA}[1]{analyze(#1)}
\newcommand{\CN}[1]{narrow(#1)}
\newcommand{\ALL}[1]{all(#1)}

\begin{table*} %[width=0.5\pagewidth, float=t]
    \begin{tabularx}{\textwidth}{l X c}
Operation & Access Path Computation & Barrier \\ \hline
\code{filter}	&	$P = S + \CA{f}$ & \\ \ENDTABLELINE
\code{map}		&	$P = \CA{\CN{f}}$  & \\ \ENDTABLELINE
\code{groupByKey}		&	$P = \ALL{I,\_1} + \REWRITE{S}{\_2.iterable.x}{\_2.x}$ & \checkmark \\ \ENDTABLELINE
\code{join}	&	$P_L = \ALL{I,\_1} + \REWRITE{S}{\_2.\_1.x}{\_2.x}$ \newline $P_R = \ALL{I,\_1} + \REWRITE{S}{\_2.\_2.x}{\_2.x}$ & \checkmark \\	\ENDTABLELINE
\code{reduce}	&	$P = \REWRITE{\CA{f}}{x}{\_2.iterable.x} +
\REWRITE{S}{\_2.x}{\_2.iterable.x}$ & \\ \ENDTABLELINE
% \code{reduce2}	&	$P = $ \newline $\REWRITE{CA(f)}{x}{\_2.iterable.x} + $ \newline $\REWRITE{S}{\_2.x}{\_2.iterable.x}$ & \\ \ENDTABLELINE 
\code{cache}	&	$P = S$ & \checkmark \\ \ENDTABLELINE
\code{save}	&	$P = \ALL{I}$ & \\ \ENDTABLELINE
\end{tabularx} 
    \caption{Access path computation and propagation for selected operations. 
    $P$ and $S$ are the access path sets of the predecessor and successor. 
    $\CA{f}$ analyzes the given closure $f$, $\CN{f}$ narrows it.
    $\REWRITE{S}{x.y}{x.z}$ rewrites the access paths in $S$ with prefix $x.y$ to have prefix $x.z$ instead.
    $I$ is the input element type of the operation, and \ALL{I} can be used to generate the input types.
    }
    \label{tbl:analysis}
\end{table*}


% make an example for map
To analyze a \code{map} operation, we need to combine the narrow closure and the
closure analysis primitive. \aos ensures that the output symbol of a closure is
always a constructor invocation. We apply the narrow closure primitive to create
a new closure, in which the new output reads from the old output only the live
fields. LMS recognizes when a field is read from a constructor invocation in the
same scope, and instead of generating an access for a field, it returns that
value directly. This happens for all the fields, therefore the old constructor
invocation will not be read anymore, and DCE will remove it. This means that the
field values, which only it was reading, will also not be read anymore, and they
too will be eliminated. When the closure has been cleaned of dead values, we can
analyze this new closure to get the access paths from it.

Table \ref{tbl:analysis} shows how relevant operations in \tool are
analysed and how access paths are propagated. The first column is the operation
name, the middle column contains the rules for propagating the access paths and
the last column shows whether we treat this operation as a barrier.
