\subsection{Operation Fusion}
\label{sec:fusion}

In Section \ref{sec:programming-model} we have shown that the \code{DColl} class
provides declarative higher-order operations. Closures passed to these
operations do not share the same scope of variables. This reduces the number of
opportunities for the optimizations described in Section
\ref{subsec:lms-optimizations}. Moreover, each data record needs to be read,
passed to and returned from the closure. In both the push and the pull data flow
model this enforces expensive virtual method calls \cite{murray_steno:_2011} for
each data record processed. To reduce this overhead and to provide more
opportunities for compiler optimizations we have implemented fusion of
the monadic operations \code{map}, \code{flatMap} and \code{filter} through the
underlying loop fusion algorithm described in Section \ref{subsec:lms-optimizations}.

The loop fusion optimization described in Section \ref{subsec:lms-optimizations}
supports horizontal and vertical fusion of loops as well as fusion of nested
loops. Also, it provides a very simple interface to the DSL developer for
specifying loop dependencies and for writing fusible loops. We decided to extend
the existing mechanism to the \code{DColl} operations although they are not
strictly loops. Alternatively, we could have taken the path of Murray et al. in
project Steno \cite{murray_steno:_2011} by generating an intermediate language which can be
used for simple fusion and code generation. Also, we could use the Coutts et al.
\cite{coutts_stream_2007} approach of converting \code{DColl} to streams and
applying equational transformation to remove intermediate results. After
implementing the algorithm by reusing the LMS loop fusion we are confident that
it required significantly less effort than other alternatives.

\begin{figure}[t]
We define the set of program \code{DColl} nodes as $D$.

For $n \in D$:
\begin{lstlisting} 
  $pred(n)$ gives the predecessor of the node in $D$
  $succ(n)$ returns a set of node successors in $D$
  $prevent\_fusion(n) =  |succ(pred(n))| > 1$
\end{lstlisting}

Lowering Transformations:
\begin{lstlisting}
  $out = map(in, op) \rightarrow  $
    $loop(shape\_dep(in, prevent\_fusion(in)), index, \{$
      $yield(out, op(iterator\_value(in)))\\$
    $\})$
  $out = filter(map, op) \rightarrow$
    $loop(shape\_dep(in, prevent\_fusion(in)), index, \{$
      $if (op(iterator\_value(in)))$
	$yield(out, iterator\_value(in)) $
    $\})$
  $out = flatMap(in, op) \rightarrow$
    $loop(shape\_dep(in, prevent\_fusion(in)), index, \{$
      $w = op(iterator\_value(in))$
      $loop(w.size, index, \{yield(out, w(index))\})$
    $\})$
\end{lstlisting}
\caption{Operation lowering transformations.}
\label{lst:lowering}
\end{figure}

Before the fusion optimization, the program IR represents an almost one to one
mapping to the operations in the programming model. Each monadic operation is
represented by the corresponding IR node which carries its data and control flow
dependencies and has one predecessor and one or more successors. On these IR
nodes we first apply a lowering transformation, which translates monadic
operations to their equivalent loop based representations. We show the program
translation for this transformation in Listing \ref{lst:lowering}.
% This
%transformation is achieved by the program translation described in Listing
%\ref{lst:lowering}. 
These rules introduce two new IR nodes: \emph{i)}
$shape\_dep(n, m)$ that carries the explicit information about its vertical
predecessor $n$ and a $prevent\_fusion$ bit $m$, and \emph{ii)}
\code{$iterator\_value$} that represents reading from an iterator of the
preceding \code{DColl}. $shape\_dep$ replaces the shape variable (e.g.
\code{in.size}) of the loop IR node. The \emph{yield} operation represents
storing to the successor collection and is used in the LMS fusion algorithm for
correct fusion of two loops.

For the back-end frameworks that \tool supports, the fusion operation is not
possible for all operations in the data-flow graph. If one node has multiple
successors, after fusion, it would form a loop that yields values to multiple
\code{DColl}s. This would be possible for Hadoop, as it supports multiple
outputs in the map phase but is not currently supported in Spark
and Crunch. To prevent fusion of such loops we added the $prevent\_fusion$ bit
to the $shape\_dep$ node. We also prevent horizontal fusion by making
$shape\_dep$ nodes always different in comparison.

After the lowering transformation we apply the loop fusion optimization from
LMS. It iteratively fuses pairs of loops for which the consumer does not have
the $prevent\_fusion$ bit set until a fixed point is reached. In each fusion
iteration all other LMS optimizations are applied as well. Afterwards, in the
code generation layer for each back-end, we specify how to compile loops with
$shape\_dep$ nodes to the most general operation (e.g. the Hadoop \code{Mapper}
class) that the framework provides. With this approach we could also generate
code directly for Hadoop MapReduce which would result in a single highly
optimized loop per \code{Mapper}. However, after doing experiments we concluded
that the gain is not significant compared to using higher level back-ends.
Therefore, as an alternative, we used the frameworks Crunch.


Unlike MapReduce based back-ends, Spark's design uses the pull data-flow model,
implemented through iterators. Generating code for the pull data-flow model from
the loop based model (push data-flow) proved to be non-trivial. After evaluating
different types of queues and array buffers we have decided to buffer
intermediate results in a 4 MB array.