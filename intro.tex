\section{Introduction}
\label{sec:introduction}

% Motivation -- three different types programming models and their problems
In the past decade, numerous systems for big data cluster computing have been
studied \cite{dean_mapreduce:_2008, yu_dryadlinq:_2008-1, olston_pig_2008-1,
thusoo_hive_2010-1, spark-nsdi}. Programming models of these systems impose a
trade-off between generality, performance and productivity. Systems like Hadoop
MapReduce \cite{hadoop}, and Dryad \cite{isard_dryad:_2007} provide a low level
general purpose programming model that allows writing fine grained and optimized
code. However, low level optimizations greatly sacrifice productivity
\cite{chambers_flumejava:_2010}. Models like Spark \cite{spark-nsdi}, FlumeJava
\cite{chambers_flumejava:_2010} and Dryad/LINQ \cite{yu_dryadlinq:_2008-1}
provide high level operations and general purpose programming models, but their
performance is limited by glue code between high level operations. Also, many
relational optimizations are impossible due to the lack of knowledge about the
program structure.

Domain-specific approaches, like Pig \cite{olston_pig_2008-1} and Hive
\cite{thusoo_hive_2010-1}, have a narrow, side effect free and
domain-specific interface that allows them to be both productive and to apply relational optimizations.
However, they come with their own set of limitations. Their programming model is
often too simple for a wide range of problems. This requires reverting to user
defined functions, which are cumbersome and again hard to optimize, or to
abandoning the model completely. Moreover, there is the high overhead of
learning a new language.
Proper tool support, like debugging tools and IDE support is often limited. It
is also hard to extend these frameworks with optimizations for a new domain.

% Existing Solutions to achieve this
Recently there have been several efforts aimed at making programming
distributed batch processing efficient, productive and general at the same time.
Steno \cite{murray_steno:_2011} implements an innovative runtime code generation
scheme that eliminates iterators in Dryad/LINQ queries. It operates over flat
and nested queries and produces a minimal number of loops without any iterator
calls. Manimal \cite{jahani_automatic_2011} and HadoopToSQL
\cite{iu_hadooptosql:_2010} apply byte code analysis, to extract information
about unused data columns and selection conditions. The reason for doing this is to gain enough
knowledge to apply common relational database optimizations for projection and
selection. However, since these solutions use static byte code analysis they
must be safely conservative which can lead to missed optimization opportunities.

% General, restricted optimizations modular/extensible portable
This paper presents \tool, a new domain-specific framework for distributed batch
processing that provides a high level declarative interface similar to
Dryad/LINQ and Spark. \tool is built upon language virtualization
\cite{moors_scala-virtualized_2012} and lightweight modular staging
\cite{rompf_lightweight_2010} (LMS), and has the same syntax and semantics as the
standard Scala programming language, with only a few restrictions.
Because \tool includes common compiler optimizations and relational optimizations, as well as
domain-specific optimizations, it produces efficient code which closely resembles hand
optimized implementations.
Furthermore, it is designed in a composable and modular way - the code generation is
completely independent of the parsing and optimizations - and allows extensions
for supported operations, optimizations and back-ends.

% Contributions
This paper makes following contributions to the state of the art:
\begin{itemize}

  \item We implement the \tool framework for distributed batch data
  processing, which provides a general high level programming model with carefully
  chosen restrictions. These restrictions allow relational, domain-specific as well as
  compiler optimizations, but do not sacrifice program generality.

  \item We introduce a novel projection insertion algorithm that operates across
  general program constructs like classes, conditionals, loops and user defined
  functions. It takes the whole program into account, and can safely apply optimizations 
  without being overly conservative due to the available effect information.

  \item We show that \tool allows easy language extension and code portability
  for distributed batch processing frameworks.

\end{itemize} 

% Sections
In Section \ref{sec:background} we provide background on LMS,
language virtualization and big data frameworks. Then, in Section
\ref{sec:programming-model} we explain the programming model and present simple
program examples. In Section \ref{sec:optimizations} we explain the novel
projection insertion optimization algorithm in Section \ref{sec:field-reduction}
and Section \ref{sec:fusion} explains the fusion optimization. We evaluate \tool
in \ref{sec:evaluation} and discuss our approach in Section
\ref{sec:discussion}. \tool is compared to state of the art in Section
\ref{sec:related-work}, future work is in Section \ref{sec:future-work} and we
conclude in Section \ref{sec:conclusion}.
