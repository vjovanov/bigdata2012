\section{Related Work}
\label{sec:related-work}

% Pig
Pig \cite{olston_pig_2008-1} is a framework for writing jobs for the Hadoop
platform using an imperative domain-specific language called Pig Latin. Pig
Latin's restricted interface allows the Pig system to apply relational
optimizations that include operator rewrites, early projection and early
filtering to achieve good performance.
It has extensive support for relational operations and allows the user to choose
between different join implementations with varying performance characteristics.
Pig Latin users need to learn a new language which is not the case with
frameworks such as Hadoop, Hive and Crunch. It does not include user defined
functions, the user needs to define them externally in another language, which
will often prevent optimizations as Pig can not analyze those. Pig Latin is not
Turing complete as it does not include control structures itself.
The language is not statically type checked so runtime failures are common and
time consuming. Also, Pig can not profit from the rich ecosystem
of productivity tools for Java.

Even though \tool also adopts a domain-specific approach, it is deeply embedded
in Scala, Turing complete and allows the user to easily define functions which
do not disable the optimizations. Currently \tool only supports one relational
optimizations. However, it includes common compiler optimizations and it is well
extensible.

% Steno
Steno \cite{murray_steno:_2011} is a .NET library that, through runtime code
generation, effectively removes abstraction overhead of the LINQ programming
model. It removes all iterator calls inside LINQ queries and provides
significant performance gains in CPU intensive jobs on Dryad/LINQ. Queries that
use Steno do not limit the generality of the programming model but optimizations
like code motion and early projection are not possible.
\tool also removes excess iterator calls from the code but during operation
fusion it enables other optimizations, especially when combined with early
projection. The drawback of \tool is that it is slightly limited in generality.

% Manimal and HadoopToSQL
Manimal \cite{jahani_automatic_2011} and HadoopToSQL \cite{iu_hadooptosql:_2010}
perform static byte code analysis on Hadoop jobs to infer different program
properties that can be mapped to relational optimizations. They both use the
inferred program properties to build indexes and achieve much more efficient
data access patterns. Manimal can additionally organize the data into columnar
storage. These approaches are limited by the incomplete program knowledge which
is lost by compilation and runtime determined functions. They both do not
restrict the programming model at all.
\tool shares the idea of providing code generality to these approaches. However,
it currently does not include data indexing schemes which could enable big
performance improvements. We believe that the full type and program information
available in \tool will enable us to build better data indexing schemes for a
larger set of user programs.
