\section{Related Work}
\label{sec:related-work}

% Pig

Pig is a framework for writing jobs for Hadoop platform using a Pig latin domain specific language. Pig latin's restricted interface allows the Pig system to apply relational optimizations that include operator rewrites, early projection and early filtering to achieve good performance on the Hadoop platform. The language is imperative and gives users slightly more leeway to achieve their tasks. It allows users to choose between different join implementations which allows user controlled optimizations for certain tasks. The programming language provides access to a general purpose programming language through UDFs.  Unlike with Hadoop, Hive and FlumeJava, Pig latin programmers have an overhead of learning the new programming language. The language is not statically type checked so runtime failures are common and time consuming. Also, pure Java approaches benefit from rich ecosystem of productivity tools. Moreover user defined functions are sometimes cumbersome to write an often disallow optimizations.

In \tool we also adopt the domain specific approach but we give user much more freedom. UDFs can be written directly in \tool and compiler and relational optimizations are not disabled by them. \tool is syntactically not a new language and it is embedded into the Scala language which gives it access to most of the productivity tools. Although we currently do not support many relational optimizations in \tool we apply aggressive compiler optimizations jointly with relational ones. \tool can be easily extended with additional optimizations without modifying the framework it self.

% Steno 
Steno is a .NET library that, through runtime code generation, effectively removes abstraction overhead of the LINQ programming model. It removes all iterator calls inside LINQ queries and provides significant performance gains in CPU intensive jobs on Dryad/LINQ. Dryad/LINQ queries that use Steno do not limit the generality of the programming model but optimizations like code motion and early projection are not possible. 
\tool also removes excess iterator calls from the code but during operation fusion it enables other optimizations, especially when combined with early projection. The drawback of \tool is that it is slightly limited in generality.    

% Manimal and HadoopToSQL
Manimal and HadoopToSQL perform static byte code analysis on Hadoop jobs to infer different program properties that can be mapped to relational optimizations. They both use the inferred program properties to build indexes and achieve much more efficient data access patterns. Manimal can additionally organize the data into columnar storage. These approaches are limited by the incomplete program knowledge which is lost by compilation and runtime determined functions. They both do not restrict the programming model at all. 
\tool is similar to these approaches in the idea of providing code generality. However, it currently does not include data indexing schemes that can give great performance improvements. We believe that in the future whole program knowledge provided by LMS will enable us to build better data indexing schemes for larger set of user programs.    
